Über 70 Prozent unserer Erde ist mit Wasser bedeckt\colorbox{yellow}{QUELLE}. Trotz bereits Jahre andauernden Explorationen sind momentan noch  \colorbox{yellow}{Prozent} des Meeresbodens unerforscht \colorbox{yellow}{QUELLE Mayer et al., 2018}. 

\begin{itemize}
	\item Jetzt darauf eingehen, warum es wichtig ist, den Meeresboden zu erforschen. https://sci-hub.hkvisa.net/10.3389/fmars.2019.00283
	\item Danach erklären, wie Meeresboden gemappt wird -> Multibeam echo-sounders und deren Definition 
	\item ...Allerdings sind Multibeam Daten nicht 100 Prozent genau und es können sich Fehler einschleichen -> Anomalien
	\item Definition Anomalien / Outlier und unterschiedliche Varianten (globale vs lokale)
	\item 
\end{itemize}


Outliers are the patterns which are not in the range of normal behavior \cite{vinuthaDetectionOutliersUsing2018} \linebreak
% https://sci-hub.hkvisa.net/10.1007/978-981-10-7563-6_53
\\
Moving window pattern! \linebreak
Chandola, V., Banerjee, A., Kumar, V.: Outlier detection: a survey\linebreak

"what techniques were used in the past and how machine learning
could help us now and in the future for a better bathymetric
data processing" \linebreak
%https://sci-hub.hkvisa.net/10.1109/OCEANSE.2019.8867321

Outlier detection (or data
cleaning) is an important step in data processing because, if an outlier data point is used
during data mining, it is likely to lead to inaccurate outputs -> 
Ramírez-Gallego, S.; Krawczyk, B.; García, S.; Wo ´zniak, M.; Herrera, F. A survey on data preprocessing for data stream mining:
Current status and future directions. Neurocomputing 2017, 239, 39–57. [CrossRef]
